\begin{abstract}
Semi-autonomous agentic systems that coordinate multiple AI agents, invoke external tools, and execute multi-step research workflows have advanced rapidly, yet the infrastructure required to persist, trace, and cumulatively extend their outputs has not kept pace. This paper synthesizes recent contributions spanning agentic architectures for scientific discovery, knowledge graph--based retrieval, empirical standards for software engineering research, methodological guidelines for LLM-based studies, and mechanistic interpretability. The synthesis reveals a structural misalignment between architectural capability and accountability: current systems either maintain structured knowledge without autonomous reasoning, or reason autonomously without enforceable provenance, and none supports cumulative, provenance-enforced research progression across sessions and documents. 

To address this divide, we propose a provenance-first architectural framework that integrates persistent knowledge graphs with multi-agent orchestration, treating span-level evidence alignment and canonicalization auditing as architectural invariants rather than post-hoc annotations. The framework contributes a reference architecture for provenance-enforcing agentic research, a faithfulness-centered evaluation framework, and a reproducibility protocol—positioning provenance-enforced knowledge infrastructure as a foundation for accountable, cumulative AI-assisted research ecosystems.
\end{abstract}