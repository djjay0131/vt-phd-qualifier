\section{Conclusion}
\label{sec:conclusion}

As semi-autonomous agentic systems move from experimental capability demonstrations into sustained research use, they expose a gap not in what these systems can produce, but in the infrastructure that governs how their outputs are traced, evaluated, and accumulated. This synthesis surfaced a consistent structural divide: current systems either maintain structured knowledge without autonomous reasoning, or reason autonomously without enforceable provenance, and none yet supports reliable research progression across sessions and documents.

We proposed a provenance-first architecture that integrates persistent knowledge graphs with multi-agent orchestration, treating span-level evidence alignment and canonicalization auditing as design constraints rather than post-hoc annotations. Together with a faithfulness-centered evaluation framework and a reproducibility protocol for agentic research, this approach positions provenance not as an optional reporting practice but as core research infrastructure.

This work does not claim to resolve the broader challenges of autonomous scientific discovery; the system remains bounded by its source literature and the capabilities of its underlying models, and it deliberately retains human oversight to balance autonomy with verifiability. Nonetheless, strengthening the connective infrastructure between claims and evidence---across time, sessions, and documents---is a prerequisite for semi-autonomous research agents that are not only capable but accountable.
